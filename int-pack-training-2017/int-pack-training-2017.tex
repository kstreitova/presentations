\documentclass{beamer}
\hypersetup{pdfpagemode=FullScreen}
\usepackage[utf8x]{inputenc}
\usepackage[czech]{babel}
\usetheme[pageofpages=of,% String used between the current page and the
                         % total page count.
          bullet=circle,% Use circles instead of squares for bullets.
          titleline=true,% Show a line below the frame title.
	  titlepagelogo=opensuse,
          alternativetitlepage=true,% Use the fancy title page.
          ]{Torino}

\setbeamerfont{title}{series=\bfseries,size=\LARGE}
\author{Tom\'{a}\v{s} Chv\'{a}tal et al.\newline {\small tchvatal@suse.com}\newline {\small Packaging/L3 - Packaging}}
\title{Basic packaging introduction for openSUSE}
\date{2017/06/23}

\AtBeginSection[]
{
	\setbeamercolor{background canvas}{bg=chameleongreen3}
	\begin{frame}[plain]
		\begin{center}\begin{huge}\textcolor{white}{\secname}\end{huge}\end{center}
	\end{frame}
	\setbeamercolor{background canvas}{bg=}
}

\AtBeginSubsection[]
{
	\setbeamercolor{background canvas}{bg=chameleongreen3}
	\begin{frame}[plain]
		\begin{center}\begin{huge}\textcolor{white}{\subsecname}\end{huge}\end{center}
	\end{frame}
	\setbeamercolor{background canvas}{bg=}
}

\begin{document}

\begin{frame}[t,plain]
\titlepage
\end{frame}

\section{Introduction}

\begin{frame}[t]{What actually is packaging}
	\begin{itemize}
	\item Milk/Amazon?
	\item More like process to deliver software stack in compact and verified form to user/customer
	\end{itemize}
\end{frame}


\begin{frame}[t]{Why do we need packages at all?}
	\begin{itemize}
	\item We need to be able to deliver software to users
  \item We need to isolate needed components
	\item We need to ensure proper testing of such software
	\item We need to compile all various software stacks together
	\item We need to provide comprehensive solutions for some tasks
    \begin{itemize}
      \item Dependencies
      \item Updating and migration
      \item User management
      \item Postinstallation configuration
    \end{itemize}
	\end{itemize}
\end{frame}


\begin{frame}[t]{Packages}
	\begin{itemize}
	\item Most operating systems have some type of packaging
    \begin{itemize}
      \item Unix/Linux
      \item Widnows
      \item Realtime operating systems
    \end{itemize}
  \item Many popular languages have their own package system/registry
    \begin{itemize}
      \item CPAN (perl)
      \item pypi (python)
      \item cabal (haskell)
    \end{itemize}
    \item Container/sandbox formats
      \begin{itemize}
        \item docker (dockerfiles, Open Containers Initiative)
        \item flatpak
        \item appimage
      \end{itemize}
	\end{itemize}
\end{frame}

\begin{frame}[t]{Packages on Linux}
	\begin{itemize}
	\item Most distributions are leveraging some package/software management
    \begin{itemize}
      \item Debian/Ubuntu (apt + dpkg)
      \item Arch linux (pacman)
      \item Gentoo linux (portage)
      \item rpm based distributions (rpm + zypper/yum/dnf)
    \end{itemize}
  \end{itemize}
\end{frame}


\begin{frame}[t]{What are packages for the user?}
	\begin{itemize}
	\item Collection of files and their respective permissions
	\item Metadata containing information about the software runtime dependencies
	\item Initial configuration provider
	\end{itemize}
\end{frame}


\begin{frame}[t]{What does it provide for the packager?}
	\begin{itemize}
	\item Tool to provide the container to the user in a unified way
	\item In our distributions the .spec file specifying various areas of what should be done
	\item Tool to verify software is buildable and distributable on various distribution codestreams
	\end{itemize}
\end{frame}

\section{Specfile surgery}

\begin{frame}[t]{Sample initial spec}
	\begin{center}\$ vim whatever.spec\end{center}
\end{frame}

\subsection{Preamble}

\begin{frame}[t]{Easy parts}
	\begin{itemize}
	\item Name, Version, Summary, Url, Patch
	\item License - SPDX \begin{small}\url{https://github.com/openSUSE/spec-cleaner\#spdx-licenses}\end{small}
    \item Group - \begin{small}\url{https://en.opensuse.org/openSUSE:Package\_group\_guidelines}\end{small}
    \item Source - with full URL path
    \item \%description
	\end{itemize}
\end{frame}

\begin{frame}[t]{BuildRequires}
	\begin{itemize}
	\item Should contain what your package needs for build
    \item Preferably it should be version limited (based i.e. on configure.ac)
    \item If something is wrong here the package should not build
	\end{itemize}
\end{frame}

\begin{frame}[t]{Requires}
	\begin{itemize}
    \item Used for Run-time dependencies
    \item Automatically populated for shared libraries
    \item Basically, all your application needs to run ought to be there
    \item If you get it wrong user might notice during runtime and not sooner so be careful
	\end{itemize}
\end{frame}

\begin{frame}[t]{Requires - scriptlets}
	\begin{itemize}
	\item Special case of require needed only for scriptlet, not during runtime of the application
    \item Used to request just something extra for the specific phase
	\end{itemize}
\end{frame}

\begin{frame}[t]{Requires - specialities}
	\begin{itemize}
    \item \%requires\_eq - for exactly same version requirement
    \item  \%requires\_ge - Translates to $>=$ on the requirement
	\end{itemize}
\end{frame}

\begin{frame}[fragile]{Example}
	\begin{small}
	\begin{verbatim}
BuildRequires: libvisio-devel >= 1.2.3
BuildRequires: cmake(GLEW) < 2.0
BuildRequires: pkgconfig(X11) => 0.9
BuildRequires: python-imaging
BuildRequires: perl(imaging)
Requires(post): update-alternatives
Requires: python-imaging
Requires: tex(yhmath)
%requires_eq perl
	\end{verbatim}
	\end{small}
\end{frame}

\begin{frame}[t]{Conflicts}
	\begin{itemize}
    \item Used to block installation of the conflicting packages
    \item Usually used in case multiple packages have and provide the same set of files
    \item All of such packages must have the Conflicts specified in them to ensure it works.
	\end{itemize}
\end{frame}

\begin{frame}[t]{Provides/Obsoletes}
	\begin{itemize}
	\item Generally used to swap one package for another
    \item Provides/Obsoletes should be always versioned
    \item Do not obsolete unless 100\% replacement
	\end{itemize}
\end{frame}

\begin{frame}[fragile]{Example}
	\begin{small}
	\begin{verbatim}
Conflicts: libwriterperfect
Provides: liboldpackage = %{version}
Obsoletes: liboldpackage < %{version}
Provides: alternativepackage = %{version}
	\end{verbatim}
	\end{small}
\end{frame}

\begin{frame}[t]{Recommends/Suggests/Supplements/Enhances}
	\begin{itemize}
	\item Weak dependencies where Recommends is soft version of Requires
    \item It is not required to have packages in this category installed contrary to Requires
    \item Supplements is mirror of Recommends, where other package states it is an addition that is to be installed if package A is present
    \item Suggest and Enhances are weaker version that do not get autoinstalled, just mentioned by zypper
	\end{itemize}
\end{frame}

\begin{frame}[fragile]{Example}
	\begin{small}
	\begin{verbatim}
# package will install wine if it gets installed
Recommends: wine
# if pulseaudio gets installed this package will as well
Supplements: pulseaudio
# if both vim and util-linux are installed the package
# will be installed too
Supplements: packageand(util-linux:vim)
# honorable mention of k3b in zypper output
Suggests: k3b
# if user installs xonotic this package will be
# mentioned by zypper
Enhances: xonotic
	\end{verbatim}
	\end{small}
\end{frame}

\begin{frame}[t]{Subpackages}
	\begin{itemize}
	\item Subpackages carry the syntax logic for the main spec package
    \item They have it's own provides/requires/scriptlets/files/etc., but BuildRequires should be at the main package for readability
    \item There are two types, appending name (ie. bla and bla-python) or completely renaming one (ie. blas.spec to provide libblas1)
    \item All subpackages must have their own Summary, Group, and \%description
	\end{itemize}
\end{frame}

\begin{frame}[fragile]{Example}
	\begin{small}
	\begin{verbatim}
%package python # generates bla-python
Summary: python bindings for bla
Group:   some/group

%description python
The python bindings providing a, b, and c for bla

%package -n format_spec_file # generates format_spec_file
Summary:        Binding replacing OBS service format_spec_file
Group:          Development/Tools/Other

%description -n format_spec_file
Alternative provider of format_spec_file functionality
	\end{verbatim}
	\end{small}
\end{frame}

\subsection{Prepare}

\begin{frame}[t]{Prepare phase}
	\begin{itemize}
	\item Phase used for source unpacking and preparations
    \item All the patches should be applied here
    \item All the source code changing operations should happen here
	\end{itemize}
\end{frame}

\subsection{Build/Check}

\begin{frame}[fragile]{Build phase}
	\begin{itemize}
	\item Phase used for source configuration and compilation
    \item The configuration step should use macros
    \begin{verbatim}%configure\end{verbatim}
    \begin{verbatim}%cmake\end{verbatim}
    \item The building should happen in threads
    \begin{verbatim}make %{?_smp_mflags}\end{verbatim}
	\end{itemize}
\end{frame}

\begin{frame}[t]{Check phase}
	\begin{itemize}
	\item Phase used for test compilation and testsuite execution
    \item If package has testsuite it must be run and all failures should be coordinated with upstream
    \item We rely on this phase to do the first round of "sanity checking" for the software stack
	\end{itemize}
\end{frame}

\subsection{Install}

\begin{frame}[t]{Install phase}
	\begin{itemize}
	\item Phase used to install the software files to the proper locations
    \item Also used to install additional files we might need to deliver (systemd units, ...)
    \item Simply just commands telling where to put what from within the compiled sources
    \item It is pretty good idea to rather use install with proper mode than cp/mkdir/\ldots
	\end{itemize}
\end{frame}

\subsection{Scriptlets}

\begin{frame}[t]{Install phase}
	\begin{itemize}
    \item Scriptlets are shell or lua scripts executed during various phases of the package install/removal or update
    \item \url{https://fedoraproject.org/wiki/Packaging:Scriptlets}
        \begin{itemize}
        \item \%pretrans - LUA only
        \item \%pre
        \item \%post
        \item \%preun
        \item \%postun
        \item \%posttrans
        \end{itemize}
	\end{itemize}
\end{frame}

\begin{frame}[fragile]{Example}
	\begin{small}
	\begin{verbatim}
Requires(post): update-alternatives

%post
update-alternatives --install %{_javadir}/el_api.jar el_api \
  %{_javadir}/%{name}-el-%{elspec}-api.jar 2030
	\end{verbatim}
	\end{small}
\end{frame}

\subsection{Filelists}

\begin{frame}[t]{Files section}
	\begin{itemize}
	\item Part of the spec stating where should be what files present among the split subpackages
    \item Can also contain exact specification for permissions/user/group they should contain
	\end{itemize}
\end{frame}

\begin{frame}[fragile]{Example}
	\begin{small}
	\begin{verbatim}
%files devel
%defattr(-,root,root)
%license COPYING
%doc doc/README.md
%config %{_sysconfdir}/%{name}.conf
%config(noreplace) %{_sysconfig}/%{name}-user.conf
%{_libdir}/*.so
%{_libdir}/pkgconfig/libwps*.pc
%{_includedir}/libwps-*
%attr(600, user, group) %{_sbindir}/secret
	\end{verbatim}
	\end{small}
\end{frame}

\subsection{Changelog}

\begin{frame}[t]{Changelog}
	\begin{itemize}
	\item The \%changelog section in openSUSE spec files is populated by build scripts
	\item We keep the content in \%{name}.changes file
    \item It should contain information that should be delivered to user about the changes
    \item The content can be pre-filled by using \texttt{osc vc} command
	\end{itemize}
\end{frame}

\begin{frame}[fragile]{Example changelog entry}
	\begin{tiny}
	\begin{verbatim}
-------------------------------------------------------------------
Tue Sep 13 12:53:23 UTC 2016 - tchvatal@suse.com

- Switch to gold, we need to use less memory when linking
- Expand constraints for the debug symbols
- Use default chromium values from master_preferences on first run
  rather than pseudo-duplicating in shellscript, bugs should be
  fixed in the masterprefs
- Add patch to fix build on 4.5+ kernels with systemlibs:
  * chromium-sandbox.patch

	\end{verbatim}
	\end{tiny}
\end{frame}


\section{Advanced topics}

\begin{frame}[fragile]{Advanced dependencies}
  \begin{itemize}
    \item Since rpm 4.13
  \end{itemize}
  \begin{small}
  \begin{verbatim}
Requires: (pkgA or pkgB or pkgC)
Requires: (pkgA or (pkgB and pkgC))
Requires: (pkgA >= 3.2 or pkgB)
Supplements: (foo and (lang-support-cz or lang-support-all))
Conflicts: (pkgA and pkgB)
Recommends: (myPkg-langCZ if langsupportCZ)
Requires: (myPkg-backend-mariaDB if mariaDB else sqlite)
	\end{verbatim}
	\end{small}
\end{frame}

\begin{frame}[t]{Library packaging}
	\begin{itemize}
	\item We require to be subpackage named as the soname they provide
    \item Must run ldconfig after install, uninstall, and update
    \item Should have proper soname data set
    \item Static libraries should be avoided unless absolutely required
	\end{itemize}
\end{frame}

\begin{frame}[fragile]{Example}
	\begin{small}
	\begin{verbatim}
%package -n libbla1
Summary: library for bla
Group: System/Libraries

%description -n libbla1
Shared library to operate with bla

%post -n libbla1 -p /sbin/ldconfig
%postun -n libbla1 -p /sbin/ldconfig

%files -n libbla1
%{_libdir}/libbla1.so.*
	\end{verbatim}
	\end{small}
\end{frame}

\begin{frame}[t]{Conditions}
	\begin{itemize}
	\item Usually used for distinguishing between various codestreams we provide
	\item \begin{tiny}\url{https://en.opensuse.org/openSUSE:Build\_Service\_cross\_distribution\_howto\#Detect\_a\_distribution\_flavor\_for\_special_code}\end{tiny}
	\item Simply conditions as in any other programming language
	\item Alternatively, once can use build conditionals
	\end{itemize}
\end{frame}

\begin{frame}[fragile]{Example}
	\begin{small}
	\begin{verbatim}
%if 0%{?something}
do stuff
%else
do other stuff
%endif
	\end{verbatim}
	\end{small}
	\begin{small}
	\begin{verbatim}
%if 0%{?suse_version} >= 1315
%bcond_without wayland
%else
%bcond_with wayland
%endif
...
%if %{with wayland}
...
%endif
	\end{verbatim}
	\end{small}
\end{frame}

\begin{frame}[fragile]{Conditions cast sample}
       \begin{small}
       \begin{verbatim}
0%{?suse_version} => "01320" => 1320
0%{?something_undefined} => "0" => 0
#
# => anything > 0 is true
%if 0%{?suse_version}
#
# => any suse with version bigger than 13.2
%if 0%{?suse_version} >= 1320
#
# => anything before 13.2?
%if (0%{?suse_version} && 0%{?suse_version} <= 1320)
       \end{verbatim}
       \end{small}
\end{frame}

\section{So how to do good packaging}

\begin{frame}[t]{Hints}
	\begin{itemize}
	\item Be lazy! Offload to upstream rather than inventing hacks
    \item Use the tools we have (osc, spec-cleaner)
    \item Inspire yourself and use the same approach to problems like the other distributions
    \item Build from Pristine Sources
    \item Document patches in changelog
    \item Source Patches are meaningful on several levels
    \item Respect the guidelines or ask if unsure about them
	\end{itemize}
\end{frame}

\section{Endnote}

\begin{frame}[t]{Further reading/Contact points}
	\begin{itemize}
	\item \url{https://duncan.codes/tutorials/rpm-packaging/}
	\item \url{https://en.opensuse.org/openSUSE:Packaging\_guidelines}
	\item \url{http://pes.suse.de/packagers}
	\item opensuse-packaging@opensuse.org
	\item pack@suse.cz
	\item \#opensuse-factory@freenode.net 
	\item \#pack@irc.suse.cz
	\end{itemize}
\end{frame}

\begin{frame}{Thanks/Questions}
	\begin{center}
	Thank you for your attention.\\
	Are there any questions?
	\end{center}
\end{frame}

\end{document}

