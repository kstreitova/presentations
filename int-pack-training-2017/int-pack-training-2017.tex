\documentclass{beamer}
\usepackage[utf8x]{inputenc}
\usepackage[czech]{babel}
\usetheme[pageofpages=of,% String used between the current page and the
                         % total page count.
          bullet=circle,% Use circles instead of squares for bullets.
          titleline=true,% Show a line below the frame title.
	  titlepagelogo=opensuse,
          alternativetitlepage=true,% Use the fancy title page.
          ]{Torino}

\setbeamerfont{title}{series=\bfseries,size=\LARGE}
\author{Tom\'{a}\v{s} Chv\'{a}tal\newline {\small tchvatal@suse.com}\newline {\small Packaging/L3 - Packaging}}
\title{Basic packaging introduction for openSUSE}
\date{2017/04/05}

\AtBeginSection[]
{
	\setbeamercolor{background canvas}{bg=chameleongreen3}
	\begin{frame}[plain]
		\begin{center}\begin{huge}\textcolor{white}{\secname}\end{huge}\end{center}
	\end{frame}
	\setbeamercolor{background canvas}{bg=}
}

\AtBeginSubsection[]
{
	\setbeamercolor{background canvas}{bg=chameleongreen3}
	\begin{frame}[plain]
		\begin{center}\begin{huge}\textcolor{white}{\subsecname}\end{huge}\end{center}
	\end{frame}
	\setbeamercolor{background canvas}{bg=}
}

\begin{document}

\begin{frame}[t,plain]
\titlepage
\end{frame}

\section{Introduction}

\begin{frame}[t]{What actually is packaging}
	\begin{itemize}
	\item Milk/Amazon?
	\item More like process to deliver software stack in compact and verified form to user/customer
	\end{itemize}
\end{frame}

\begin{frame}[t]{Why do we need packages at all?}
	\begin{itemize}
	\item We need to be able to deliver software to users
	\item We need to ensure proper testing of such software
	\item We need to compile together all various software stacks together
	\item We need to provide comprehensive solutions for some tasks (postinst configuration)
	\end{itemize}
\end{frame}

\begin{frame}[t]{What are packages for the user?}
	\begin{itemize}
	\item Collection of files and their respective permissions
	\item Metadata containing information about the software runtime dependencies
	\item Intial configuration provider
	\end{itemize}
\end{frame}


\begin{frame}[t]{How about what is it for the packager?}
	\begin{itemize}
	\item Tool to provide the container to the user in unified way
	\item In our distributions the .spec file specifying various areas of what should be done
	\item Tool to verify software is buildable and distributable on various distribution codestreams
	\end{itemize}
\end{frame}

\section{Specfile surgery}

\begin{frame}[t]{Sample initial spec}
	\begin{center}\$ vim whatever.spec\end{center}
\end{frame}

\subsection{Preamble}

\begin{frame}[t]{Easy parts}
	\begin{itemize}
	\item Name, Version, Summary, Url
    \item License - SPDX formatted
    \item Source - with full URL path
    \item \%description
	\end{itemize}
\end{frame}

\begin{frame}[t]{BuildRequires}
	\begin{itemize}
	\item Should contain what your package needs for build
    \item Preferably it should be version limited (based i.e. on configure.ac)
    \item If something is amiss here the package should not build
	\end{itemize}
\end{frame}

\begin{frame}[t]{Requires}
	\begin{itemize}
    \item Used for Run-time dependencies
    \item Automatically populated for shared libraries
    \item Basically all your application needs to run ought to be there
    \item If wrong one will notice when using the application. It is tricky so be careful
	\end{itemize}
\end{frame}

\begin{frame}[t]{Requires - scriptlets}
	\begin{itemize}
	\item Special case of require needed only for scriptlet not during runtime
    \item Used to request just something extra for the phase
    \item Alternatively also used to ensure something be installed
	\end{itemize}
\end{frame}

\begin{frame}[t]{Requires - specialities}
	\begin{itemize}
    \item \%requires\_eq - for exactly same version requirement
    \item  \%requires\_ge - Translates to >= on the requirement
	\end{itemize}
\end{frame}

\begin{frame}[fragile]{Example}
	\begin{small}
	\begin{verbatim}
BuildRequires: libvisio-devel >= 1.2.3
BuildRequires: cmake(GLEW) < 2.0
BuildRequires: pkgconfig(X11) => 0.9
BuildRequires: python-imaging
Requires(post): update-alternatives
Requires: python-imaging
%requires_eq perl
	\end{verbatim}
	\end{small}
\end{frame}

\begin{frame}[t]{Conflicts}
	\begin{itemize}
    \item Used to block installation of the conflicting packages
    \item All requiring packages should have the conflict on each other
	\end{itemize}
\end{frame}

\begin{frame}[t]{Provides/Obsoletes}
	\begin{itemize}
	\item Generally used to swap one package for another
    \item Provides/Obsoletes should be always versioned
    \item Do not obsolete unless 100\% replacement
	\end{itemize}
\end{frame}

\begin{frame}[fragile]{Example}
	\begin{small}
	\begin{verbatim}
Conflicts: libwriterperfect
Provides: liboldpackage = %{version}
Obsoletes: liboldpackage < %{version}
Provides: alternativepackage = %{version}
	\end{verbatim}
	\end{small}
\end{frame}

\begin{frame}[t]{Subpackages}
	\begin{itemize}
	\item Subpackages carry the syntax logic for the main spec package
    \item They have it's own provides/requires/scriptlets/files/etc., but buildrequires should be at main package for readability
    \item There are two types, appending name (ie. bla and bla-python) or completely renaming one (ie. blas.spec to provide libblas1)
	\end{itemize}
\end{frame}

\begin{frame}[fragile]{Example}
	\begin{small}
	\begin{verbatim}
%package python # generates bla-python
Summary: python bindings for bla
Group:   some/group

%description python
The python bindings providing a, b, and c for bla
	\end{verbatim}
	\end{small}
\end{frame}

\subsection{Prepare}

\begin{frame}[t]{Prepare phase}
	\begin{itemize}
	\item Phase used for source unpacking and preparations
    \item All the patches should be applied here
    \item All the source code changing operations should happen here
	\end{itemize}
\end{frame}

\subsection{Build/Check}

\begin{frame}[fragile]{Build phase}
	\begin{itemize}
	\item Phase used for source configuration and compilation
    \item The configuration step should use macros (ie. \begin{verbatim}%configure\end{verbatim} or \begin{verbatim}%cmake\end{verbatim})
    \item The building should happen in threads (ie. \begin{verbatim}make %{?_smp_mflags\end{verbatim}")
	\end{itemize}
\end{frame}

\begin{frame}[t]{Check phase}
	\begin{itemize}
	\item Phase used for test compilation and testsuite execution
    \item If package has testsuite it must be run and all failures should be coordinated with upstream
    \item We rely on this phase to do the first round of "sanity checking" for the software stack
	\end{itemize}
\end{frame}

\subsection{Install}

\begin{frame}[t]{Install phase}
	\begin{itemize}
	\item Phase used to install the software files to the proper locations
    \item Also used to install additional files we might need to deliver (systemd units, ...)
    \item Simply just commands telling where to put what from within the compiled sources
	\end{itemize}
\end{frame}

\subsection{Scriptlets}

\begin{frame}[t]{Install phase}
	\begin{itemize}
    \item Scriptlets are shell or lua scripts executed during various phases of the package install/removal or update
    \item \url{https://fedoraproject.org/wiki/Packaging:Scriptlets}
        \begin{itemize}
        \item \%pretrans - LUA only
        \item \%pre
        \item \%post
        \item \%preun
        \item \%postun
        \item \%posttrans
        \end{itemize}
	\end{itemize}
\end{frame}

\begin{frame}[fragile]{Example}
	\begin{small}
	\begin{verbatim}
Requires(post): update-alternatives
%post
update-alternatives --install %{_javadir}/el_api.jar el_api \
  %{_javadir}/%{name}-el-%{elspec}-api.jar 2030
	\end{verbatim}
	\end{small}
\end{frame}

\subsection{Filelists}

\begin{frame}[t]{Files section}
	\begin{itemize}
	\item Part of the spec stating where should be what files present among the split subpackages
    \item Can also contain exact specification for permissions/user/group they should contain
	\end{itemize}
\end{frame}

\begin{frame}[fragile]{Example}
	\begin{small}
	\begin{verbatim}
%files devel
%defattr(-,root,root)
%{_libdir}/*.so
%{_libdir}/pkgconfig/libwps*.pc
%{_includedir}/libwps-*
	\end{verbatim}
	\end{small}
\end{frame}

\section{Advanced topics}

\begin{frame}[t]{Library packaging}
	\begin{itemize}
	\item We require to be subpackage named as the soname they provide
    \item Must run ldconfig after install and uninstall, updates count
    \item Should have proper soname data set
    \item Never consider packaging static library
	\end{itemize}
\end{frame}

\begin{frame}[fragile]{Example}
	\begin{small}
	\begin{verbatim}
%package -n libbla1
Summary: library for bla
Group: System/Libraries

%description -n libbla1
Shared library to operate with bla

%post -n libbla1 -p /sbin/ldconfig
%postun -n libbla1 -p /sbin/ldconfig

%files -n libbla1
%{_libdir}/libbla1.so.*
	\end{verbatim}
	\end{small}
\end{frame}

\begin{frame}[fragile]{Conditions}
	\begin{itemize}
	\item Usually used for distinguishing between various codestreams we have
	\item Simply conditions as in any other programming language
	\end{itemize}
	\begin{small}
	\begin{verbatim}
%if 0%{?something}
do stuff
%else
do other stuff
%endif
	\end{verbatim}
	\end{small}
\end{frame}

\begin{frame}[fragile]{Conditions cast sample}
       \begin{small}
       \begin{verbatim}
%define suse_version 1320
#
# rpm casts that into a number
0%{?suse_version} => "01320" => 1320
0%{?something_undefined} => "0" => 0
#
# => anything > 0 is true
%if 0%{?suse_version}
#
# => any suse with version bigger than 13.2
%if 0%{?suse_version} >= 1320
#
# => anything before 13.2?
%if (0%{?suse_version} && 0%{?suse_version} <= 1320)
       \end{verbatim}
       \end{small}
\end{frame}

\section{So how to do good packaging}

\begin{frame}[t]{Hints}
	\begin{itemize}
	\item Be lazy! Offload to upstream and inspire by other distributors.
    \item Use the tools we have (osc, spec-cleaner)
    \item Inspire yourself and use same approach to problems like the others
    \item Build from Pristine Sources
    \item Document patches in changelog
    \item Source Patches are meaningful on several levels
    \item Respect the guidelines or ask if unsure
	\end{itemize}
\end{frame}

\section{Endnote}

\begin{frame}[t]{Further reading/Contact points}
	\begin{itemize}
	\item \url{https://en.opensuse.org/openSUSE:Packaging\_guidelines}
	\item \url{http://pes.suse.de/packagers}
	\item opensuse-packaging@opensuse.org
	\item pack@suse.cz
	\item \#opensuse-factory@freenode.net 
	\item \#pack@irc.suse.cz
	\end{itemize}
\end{frame}

\begin{frame}{Thanks/Questions}
	\begin{center}
	Thank you for your attention.\\
	Are there any questions?
	\end{center}
\end{frame}

\end{document}

